\section{Introduction}
The Reverse Cuthill McKee (RCM) is an algorithm to permute a sparse matrix that has a symmetric sparsity pattern into a band matrix form with a small bandwidth. RCM is an alternated form of the Cuthill–McKee algorithm (CM). Both algorithms generate an R set of vertices, that represents the new order of the final graph's vertices. Their difference is that RCM reverses the resulting indexes in R. As we can see in Figure \ref{fig:d1}, the final constructed graph is identical with the initial and their only difference is the node labelling. \par
\bigbreak

